\section{Discussion and Outlook}

\begin{figure}[t]
\centering
\includegraphics[width=8cm]{figures/discussion_overview}
\caption{Rule learning using external sources.}
\label{fig:discussion_overview}
\end{figure}

In this tutorial, we discussed the different approaches for rule induction and reasoning over real-life KGs. We briefly illustrated the common method for KG construction and the incompleteness and inaccuracy challenges they face. We also illustrated the difference between \textit{Horn} and  \textit{nonmonotonic} logic program and their connection to association rules. Later, we reviewed some of the existing systems for constructing Horn rules from KGs and the proposed rule evaluation measurements under OWA. Furthermore, we discussed the exiting inductive and abductive methods for learning nonmonotonic rules and their applicability over real-life KGs. Finally, we discussed rule revision approach RUMIS, which tries to capture exceptional cases to achieve accurate and consistent predictions. 

Despite the advances in rule learning, the existing methods still learn limited formats for rules, \eg closed rules, with even restrictive language bias. Additionally, they are still prone to KG incompleteness and not being able to determine the possible gaps in the data under OWA. These challenges lead to the generation of uninteresting and noisy rules. Follows some possible directions to overcome such challenges. 


\leanparagraph{Rule Learning with External Sources}  While the rule construction methods may vary for the various approaches ~\cite{amie,op,rdf2rules}, the result quality heavily depends on the utilized evaluation metric. Various rule measures have been introduced. Nevertheless, most of them are computed based on only the given graph, and cover only a small subset of local patterns in the KG, thus might wrongly estimate the quality of extracted rules since real-world KGs are usually highly incomplete.

%\section{Discussion and Outlook}
%\label{sec:discussion_outlook}
%Toward the rule-based KG completion problem, a number of future directions could be put into consideration.
%\subsubsection{Rule Learning with External Source.}

%In most of rule learning systems have been proposed \cite{amie,op,rdf2rules}, while their rule construction methods may vary, the core of them is at the proposed rule evaluation metric. Various rule measures have been introduced, from the simplest to the most sophisticated one. Nevertheless, most of them are computed based on only the given graph, and cover only a small subset of local patterns in the KG, thus might wrongly estimate the quality of extracted rules since real-world KGs are usually highly incomplete.
%
%One promising possibility to tackle this problem is to incorporate external related data from outside of the KG. The overview of such rule learning system could be described in Figure \ref{fig:discussion_overview}. The KG related external data can be extracted from many sources (e.g. crowd-sourcing, Web-extraction), and is obviously useful not only for rule evaluation, but also for rule construction over the KG. For instance, external data can give some feedback about the quality of predicted facts in several forms such as binary decision (\ie true or false) or a likelihood score. This feedback could be then taken into account for rule quality evaluation or exception capturing. In addition, external KG meta-data could gives some information about the existence of certain types of facts within the KG (as exploited in CARL~\cite{carl}). Moreover, we can also somehow learn rules directly from the text and then apply them back to the KG.
%
%An alternative relational learning method for KG completion is to learn representations (i.e. embeddings) of entities and relations for predicting likelihood of unseen facts. While these methods capture global patterns in the data, the predictions that they produce are not interpretable~\cite{Shakerin2018}. Many such models are been proposed \cite{Wang2017}, in which some of them could even be extended with additional unstructured knowledge (e.g., text corpus). Hence, integrating these embedding models into the rule learning approach might be a potential solution for the problem of rule-based KG completion.
%\subsubsection{Learning Various Rule Forms}
%In the KG completion problem, as mentioned, most of the rule mining approaches only mine $closed$ rules. Nevertheless, other form of rules might be also interesting. For example, rules with disjunction (e.g. $isMale(X) \vee isFemale(X) \leftarrow isPerson(X)$), rules with quantifier (e.g. $(\exists Y: playsInstrument(X, Y)) \leftarrow isMusician(X))$). Even though these kinds of rule do not directly make predictions on the knowledge graph since we do not know exactly which facts of the head are true, they might give us some useful constraints about the knowledge graph.
