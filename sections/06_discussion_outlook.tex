\section{Discussion and Outlook}\label{sec:disc}

\begin{figure}[t]
\centering
\includegraphics[width=8cm]{figures/discussion_overview}
\caption{Rule learning with external sources}
\label{fig:discussion_overview}
\end{figure}

In this tutorial, we have presented a brief overview of the current techniques on rule induction from knowledge graphs and have demonstrated how reasoning over KGs using the learned rules  can be exploited for KG completion. % discussed the different approaches for rule induction and reasoning over real-life KGs. We briefly illustrated the common method for KG construction and the incompleteness and inaccuracy challenges they face. We also illustrated the difference between \textit{Horn} and  \textit{nonmonotonic} logic program and their connection to association rules. Later, we reviewed some of the existing systems for constructing Horn rules from KGs and the proposed rule evaluation measurements under OWA. Furthermore, we discussed the exiting inductive and abductive methods for learning nonmonotonic rules and their applicability over real-life KGs. Finally, we discussed rule revision approach RUMIS, which tries to capture exceptional cases to achieve accurate and consistent predictions. 

While the problem of rule-based KG completion has recently gained an increasing attention, several interesting research directions are still left unexplored. % Despite the advances in rule learning, the existing methods still learn limited formats for rules, \eg closed rules, with even restrictive language bias.
% Additionally, they are still prone to KG incompleteness and not being able to determine the possible gaps in the data under OWA. These challenges lead to the generation of uninteresting and noisy rules. Follows some possible directions to overcome such challenges. 


\leanparagraph{Rule learning with external sources}  % One of the main difference between the existing approaches is the rule evaluation metric which affects the quality of the resulting rule set heavily. However,  existing measures only consider the given KG in their computation, and cover a small subset of the local patterns as well. Other missing facts are not counted, therefore, the estimated quality of rules might be inaccurate.
One promising direction is to consider pieces of evidence from other external resources while extracting the rules from KGs and estimating their quality %of the rules as shown in 
(see Figure~\ref{fig:discussion_overview}). This external resource can be of a different nature ranginf from %either be 
a human expert giving feedback about the correctness of %the 
a given rule %or its predictions 
as done in~\cite{Dzyuba2017} in the context of frequent intemset mining, %or a community voting via crowd-sourcing, 
or a dedicated fact-checking engine that given a fact, e.g., $\mi{bornIn(einstein,ulm)}$ relies on the existing Web resources for estimating its truthfulness %over textual resources such as 
This amounts to utilizing such automated fact-checking systems as Defacto~\cite{defacto} or FactChecker~\cite{factchecker} in the rule induction process. % In addition, external KG meta-data could give some information about the existence of certain types of facts within the KG (as in CARL~\cite{carl}).

 An alternative relational learning method for KG completion is to learn representations (i.e. embeddings) of entities and relations from a given KG possibly enriched with additional information sources (e.g., text) with the % propose
purpose of estimating the likelihood of unseen facts  % Recently, several models have been proposed, while many of them achieved reasonable results quality~
(see \cite{Wang2017} for overview). However, %yet 
the
%ir 
predictions produced by such approaches are not interpretable~\cite{Shakerin2018}. % Furthermore, some of them allow integrating unstructured resources during the training phase. Utilizing such models in the rule learning process can achieve promising results.% as shown in~\cite{thinh_tr}.
Interlinking these statistical methods with inductive rule learning is a promissing further direction.

\leanparagraph{Learning other rule forms}
%Prominent rule learning approaches only extract normal $closed$ logical rules, which may not capture all interesting patterns. However, 
Inducing rules of  advanced forms, e.g., %other rule forms such as rules 
with disjunction (\eg $male(X) \vee female(X) \leftarrow person(X)$) \ds{This rule is too simple, can be learned using itemsets, example with binary facts atoms is needed} or rules with existential quantifiers in the head (\eg $\mi{\exists Y: playsInstrument(X, Y) \leftarrow musician(X),bandMember(X,Z)}$) is an interesting topic to study. Several recent works on mining keys in KGs \cite{vickey,DBLP:conf/www/LajusS18} and detecting mandatory relations are relevant, but they do not directly address the mentioned rule forms. A combination of techniques from relational rule learning \cite{DBLP:books/daglib/0021868} and propositionalization approaches \cite{propos} can be made use of, yet it is unclear how to detect KG parts that are worth propositionalizing; the details are to be explored. % can encode more interesting patterns
% Even though these
While rules with existential variables in the head may not directly infer new facts, they may reflect surprising patterns in the data and, moreover,  can be used as a guidance for the information retrieval approaches.

Rules that reflect correlations between edge counts in KGs such as \emph{``If you have two siblings then you parents are likely to have 3 children''} have been studied in \cite{carl}. Inducing more general rules, reflecting mathematical functions one edge counts is still an open problem, which due to the large search space of possible hypothesis is far from trivial.

% \leanparagraph{Learning temporal rules} 
% Current approaches extract the rules from the directly stated entities and relations. More promising direction would be to learn patterns over more unstated relations such as temporal rules. For examples, KGs usually contain the dates of the events, \eg $\mi{dateOfbirth}$ or $happenedOn$ which is sparse data and requires arithmetic operator to make use of them, \eg $before(X,Y)$, $after(X,Y)$. Learning interesting rules over this data is still not well studied.
Learning temporal rules and constraints such as \emph{``You cannot graduate from a university before being born''} is another promissing future work direction. Deductive reasoning over temporal KGs has been recently considered in \cite{DBLP:conf/aaai/ChekolPSS17}; however the inductive setting has not yet been studied in full details to the best of our knowledge.

\leanparagraph{Hybrid rule learning techniques} Inducing rules from text by exploiting natural language processing (NLP) techniques has been studied in several works \ds{Insert citations}. Combining these linguistics-based methods with inductive rule learning from KGs has a great potential. Indeed, the number of predicates in most encyclopedic KGs is rather limited, since they are primarily extracted from Wikipedia; this prohibits the extraction of many interesting rule patterns such as ``\ds{TODO: insert. The previous rule about injured people has a negative flavor for being in the end of the article and it needs more motivation}''. %, like % where such non-standard facts as ``Einstein worked with Feynmen'' are rarely added. The number of predicates is limited in most of the KGs. Therefore, sometimes it is hard to learn interesting patterns from the KG only. In contrast, textual resources are rich with verbal phrases that indicate events or relations. However, learning the simplest rules from unstructured resources is challenging. Therefore, it worth investigating constructing hybrid rules that combine both resources.  For example, to learn a rule such as

% \begin{align*}
% \mi{participatedIn(X,Y)} \leftarrow & \mi{playsFor(X,Z)}, \mi{competes(Z,Y)},\\ & \naf injuredDuring(X,Y)
% \end{align*}
% \ds{would suggest to add a different rule here, this }
% The rule encodes that a player participates in a championship if he plays for a competing team unless he was injured during the championship. Learning the positive part of the rule can be easily done over the KG, yet the last one is more common to find in the text. 




%\section{Discussion and Outlook}
%\label{sec:discussion_outlook}
%Toward the rule-based KG completion problem, a number of future directions could be put into consideration.
%\subsubsection{Rule Learning with External Source.}

%In most of rule learning systems have been proposed \cite{amie,op,rdf2rules}, while their rule construction methods may vary, the core of them is at the proposed rule evaluation metric. Various rule measures have been introduced, from the simplest to the most sophisticated one. Nevertheless, most of them are computed based on only the given graph, and cover only a small subset of local patterns in the KG, thus might wrongly estimate the quality of extracted rules since real-world KGs are usually highly incomplete.
%
%One promising possibility to tackle this problem is to incorporate external related data from outside of the KG. The overview of such rule learning system could be described in Figure \ref{fig:discussion_overview}. The KG related external data can be extracted from many sources (e.g. crowd-sourcing, Web-extraction), and is obviously useful not only for rule evaluation, but also for rule construction over the KG. For instance, external data can give some feedback about the quality of predicted facts in several forms such as binary decision (\ie true or false) or a likelihood score. This feedback could be then taken into account for rule quality evaluation or exception capturing. In addition, external KG meta-data could gives some information about the existence of certain types of facts within the KG (as exploited in CARL~\cite{carl}). Moreover, we can also somehow learn rules directly from the text and then apply them back to the KG.
%
%An alternative relational learning method for KG completion is to learn representations (i.e. embeddings) of entities and relations for predicting likelihood of unseen facts. While these methods capture global patterns in the data, the predictions that they produce are not interpretable~\cite{Shakerin2018}. Many such models are been proposed \cite{Wang2017}, in which some of them could even be extended with additional unstructured knowledge (e.g., text corpus). Hence, integrating these embedding models into the rule learning approach might be a potential solution for the problem of rule-based KG completion.
%\subsubsection{Learning Various Rule Forms}
%In the KG completion problem, as mentioned, most of the rule mining approaches only mine $closed$ rules. Nevertheless, other form of rules might be also interesting. For example, rules with disjunction (e.g. $isMale(X) \vee isFemale(X) \leftarrow isPerson(X)$), rules with quantifier (e.g. $(\exists Y: playsInstrument(X, Y)) \leftarrow isMusician(X))$). Even though these kinds of rule do not directly make predictions on the knowledge graph since we do not know exactly which facts of the head are true, they might give us some useful constraints about the knowledge graph.
