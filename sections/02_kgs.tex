\section{Knowledge Graphs(1.5 pages)}
\label{sec:kgs}
Knowledge Graphs (KGs) are large collections of information modeled in the form of entities (such as people, organizations, or places) and the relationships between them. KGs were introduced in the Semantic Web community to create the "web of data" which is readable by machines. This was pushed forward by introducing the W3C Resource Description Framework (RDF)~\cite{rdf2004}. Under this representation entities in the KG are usually encoded as the nodes of the graph and the relations as a directed edges between them~\cite{Nickel2015ARO}. In literature, a simplified representation of KGs as sets of triples in the form of$\tuple{subject, predicate,object}$ (SPO triples, for short) is used. Alternatively, they can also be represented as binary predicate $p(s,o)$, where $subject$ and $object$ are the arguments. 

\begin{example}
The statement \textit{``Christopher Nolan is the director of the science fiction movie Interstellar, which won the Best Visual Effects Oscar"} will be represented in several SPO triples:\\
$\tuple{chris\_nolan, isA, director}$,\\
$\tuple{chris\_nolan, directed, interstellar}$,\\
$\tuple{interstellar, gerne, science\_fiction}$, and \\
$\tuple{interstellar, wonPrize, visual\_effects\_oscar}$.
\end{example}





%about people, organizations, places, etc. Predicates encodes the relations between them, \eg $\tuple{einstein, wasBornIn, Ulm}$. KGs can also be represented similar to formal logic syntax as binary predicate $p(s,o)$, where $subject$ and $object$ are the arguments. KGs can be seen as instantiations for some ontological schema.  
%\leanparagraph{Definition}

\leanparagraph{Construction} KGs are constructed either manually via crowd-sourcing as for \eg Freebase~\cite{Freebase} and Wikidata~\cite{wikidata}, or automatically extracted from semi-structured resources such as Wikipedia, like  YAGO~\cite{yago} and DBpedia~\cite{dbpedia}, or from unstructured resources resulting in KGs such as  NELL~\cite{NELL-aaai15}, KnowledgeVault~\cite{KnowledgeVault}. Table~\ref{kgs}, shows examples for several available KGs (see~\cite{Nickel2015ARO,DBLP:journals/semweb/Paulheim17} for more information).




\leanparagraph{}

\leanparagraph{Examples and Statistics}
