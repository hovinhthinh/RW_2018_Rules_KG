\section{Knowledge Graphs}
\label{sec:kgs}

% \subsection{Graph-based Representation}
Knowledge Graphs (KGs) have been introduced in the Semantic Web community to create the ``Web of data'' that can be readable by machines. They represent interlinked collections of factual information, and  are often encoded using the RDF data model \cite{rdf2004}. This data model represents the content of the graph with a set of triples of the form $\tuple{\mi{subject\;predicate\;object}}$ corresponding to positive unary and binary first-order logic (FOL) facts.  % are large collections of information modeled in the form of entities (such as people, organizations, or places) and the relationships between them.
%KGs %were 
%is
%can be . % This was pushed forward by introducing the W3C Resource Description Framework (RDF)~\cite{rdf2004}.
%Under this representation, the entities are encoded as the nodes of the graph and the relations as directed edges between them
%~\cite{Nickel2015ARO}. 
%In various literature, for simplicity KGs are represented as sets of triples in the form of $\tuple{subject, predicate, object}$ (SPO triples, for short). Alternatively, others adopt a factual representation for which a KG $\cG$ is defined over the signature $\Sigma_{\cG}=\tuple{\cR,\cC}$, where $\cR$ is the set of binary predicates (aka relations) and $\cC$  is the set of constants (aka entities) appearing in $\cG$.  
%\gad{could you please check this KG signature def?}\ds{please fix typos, otherwise it is ok}

Formally, we assume countable sets $\cR$ of unary and binary predicate names and $\cC$ of constants. A knowledge graph $\cG$ is a finite set of groud atoms $a$ of the form $p(s,o)$ and $c(s)$ over $\cR \cup \cC$. With $\Sigma_{\cG}$, the signature of $\cG$, we denote elements of $\cR \cup \cC$ that occur in $\cG$.

\begin{example} Figure~\ref{rdf} shows a snippet of a graph about people, marriage %their 
relations between them %to each other 
and  their %places of 
living places. For instance, the upper half encodes the information that ``Ann has a brother John, and lives with her husband Brad in Berlin, which is a metropolitan.''. 
%while her brother John lives in Chicago with his wife Kate. Moreover, both Berlin and Chicago are metropolitans'', which %would also 
This knowledge is % also 
represented by the following set of FOL facts $\mi{\{hasBrother(ann, john), livesIn(ann, berlin)}$, $\mi{livesIn(brad, berlin)}$, $\mi{isMarriedTo(brad, ann)}$
% $\mi{hasBrother(ann, john)}$, \\$\mi{livesIn(john, chicago)}$, $\mi{livesIn(kate, chicago)} \}$, $\mi{isMarriedTo(john, kate)}$,
$\mi{metropolitan(berlin)}\}$. For the given KG snippet $\cG$ the set $\cR$ contains $\mi{livesIn,isMarriedTo,hasBrother,hasFriend,}$\\$\mi{researcher,metropolitan,artist}$, while the set $\cC$ comprises of the names of the people and the locations depicted on Figure~\ref{rdf}. \qed
\end{example}
%\begin{example}
%The statement \textit{``Christopher Nolan is the director of the science fiction movie Interstellar, which won the Best Visual Effects Oscar"} will be represented in several SPO triples:\\
%$\tuple{chris\_nolan, isA, director}$,\\
%$\tuple{chris\_nolan, directed, interstellar}$,\\
%$\tuple{interstellar, gerne, science\_fiction}$, and \\
%$\tuple{interstellar, wonPrize, visual\_effects\_oscar}$.
%\end{example}





%about people, organizations, places, etc. Predicates encodes the relations between them, \eg $\tuple{einstein, wasBornIn, Ulm}$. KGs can also be represented similar to formal logic syntax as binary predicate $p(s,o)$, where $subject$ and $object$ are the arguments. KGs can be seen as instantiations for some ontological schema.  
%\leanparagraph{Definition}

%\subsection{Knowledge Graph Construction and Quality}
% Please add the following required packages to your document preamble:
% \usepackage{booktabs}
\begin{table}[t]
\centering


\begin{tabular}{@{}llll@{}}
\toprule
Knowledge Graphs       & \# Entities & \# Relations & \# Facts \\ \midrule
DBpedia (en)%\footnote{as reported by~\cite{DBLP:journals/semweb/Paulheim17} on 2015 version}               
					   &      4.8 M  &       2800       &   176 M       \\
Freebase			   &     40 M    &      35000   &    637 M \\
YAGO3%\footnote{\url{https://www.mpi-inf.mpg.de/departments/databases-and-information-systems/research/yago-naga/yago/statistics/}}                 
 &      3.6 M   &       76   &   61 M  \\
Wikidata%\footnote{\url{https://tools.wmflabs.org/wikidata-todo/stats.php}}       
&      46 M   &       4700   &   411 M  \\
%WordNet                &             &              &          \\
Google Knowledge Graph &      570 M  &       35000  &   18000 M\\ \bottomrule
\end{tabular}
\caption{Examples of real-world KGs and their statistics~\cite{Nickel2015ARO,DBLP:journals/semweb/Paulheim17}.}
\label{tab:kgs}

\end{table}
All approaches for KG construction can be roughly classified into two major groups: manual and (semi-)automatic. The examples of KGs constructed manually include, e.g., WordNet~\cite{wordnet} which has been created by a group of experts or Freebase~\cite{Freebase} and Wikidata~\cite{wikidata} constructed collaboratively by volunteers. 
Automatic population of KGs from semi-structured resources such as Wikipedia info-boxes using regular expressions and other techniques gave rise to such KGs as YAGO~\cite{yago} and DBpedia~\cite{dbpedia}. % More eager approaches 
There is also a line of projects %propose 
devoted to the extraction of facts from unstructured resources using natural language processing and machine learning methods. %resulting in KGs such as  
For example, NELL~\cite{nell} and KnowledgeVault~\cite{KnowledgeVault} belong to this category. Table~\ref{tab:kgs}, shows examples for several popular KGs and their statistics. %can be found in~\cite{Nickel2015ARO,DBLP:journals/semweb/Paulheim17}). 

%\gad{should we speak about the sizes and scalability of existing approaches}\ds{reporting statistics is enough}

While the existing KGs contain millions of facts, they are still far from being complete, and therefore treated under the open world assumption, i.e., facts not present in them are treated as unknown rather then false. For example, given the KG in Figure~\ref{rdf} we cannot say whether $\mi{livesIn(dave,chicago)}$ is true or false. This is opposed to the closed world assumption made in databases, under which $\mi{\neg livesIn(dave,chicago)}$ would be inferred from the given data.

Along with completeness, the quality of KGs can also be %is normally 
measured along several other dimensions such as size and correctness.  We refer the reader to~\cite{Nickel2015ARO,DBLP:journals/semweb/Paulheim17} for further discussions on the available KGs and their quality.

% Both manually and automatically constructed KGs are unfortunately still far from modeling the real world in two aspects: (i) they are still \textit{incomplete} with relatively \textit{low coverage} for the existing real-world entities and the possible inter-entities relationships. (ii) they have \textit{inaccurate} facts resulting from the extraction errors and the heuristic methods applied while construction as well.  



% \subsection{Open World Assumption}

% Existing KGs consist of positive facts only (\ie true relations). There are two paradigms to interpret the missing fact: (i) under \textit{closed world assumption} (CWA), in which missing edges indicate a false relationship. (ii) under open world assumption (OWA), such that non-existing relations are interpreted as unknown and can be either true or false. For example, in Figure~\ref{rdf}, the missing \textit{livesIn} link between \textit{Dave} and \textit{Chicago} would be interpreted as $\neg livesIn(dave, chicago)$ under CWA. However, under OWA, no conclusion can be made. Given the \textit{incompleteness} and \textit{inaccuracy} of the existing real-life KGs, they are usually interpreted under OWA~\cite{Nickel2015ARO}. 

%To study the incompleteness of KGs, we assume the existence of an ideal graph $\cG^i$ containing a snapshot of the current world; hence, we define the incompleteness of knowledge graphs as:

Suppose we had an \emph{ideal} KG $\cG^i$ that contains all correct and true facts in the world that reflect the relations from $\cR$ that hold among the entities in $\cC$. Then the gap between $\cG$ and its ideal version $\cG^i$ can be defined as follows \cite{rdfcomp}: 

\begin{definition}[Incomplete Knowledge Graph] An incomplete KG is a pair
    $G = (\cG, \cG^i)$ of two KGs, where $\cG\subseteq \cG^i$ and
    $\Sigma_{\cG}=\Sigma_{\cG^i}$. We call $\cG$ the available
    graph and $\mathcal{G}^i$ the ideal graph.  \end{definition}
    
Note that $\cG^i$ is an abstract construct, which is normally unavailable. The KG completion task concerns the reconstruction of this KG given the available data and possibly other external information sources.

%This definition allows evaluating the proposed approaches for completing the knowledge graph.  

